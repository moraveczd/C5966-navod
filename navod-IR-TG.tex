\documentclass[12pt]{article}

\usepackage[a4paper]{geometry}
\usepackage[absolute,overlay]{textpos}
\usepackage{graphicx}
\usepackage{tabularx}
\usepackage{hyperref}
\usepackage{adjustbox}
\usepackage{titling}

\usepackage[version=4]{mhchem}

\newcommand{\subtitle}[1]{%
	\posttitle{%
		\par\end{center}
	\begin{center}\large#1\end{center}
	\vskip0.5em}%
}

\usepackage{fontspec}
\usepackage{unicode-math}

\usepackage{polyglossia}
\setdefaultlanguage{czech}

\def\uv#1{„#1“}

\title{C5966 Vybrané analytické metody a techniky konzervace - cvičení}

\subtitle{Infračervená spektroskopie a termická analýzy\\ \url{https://is.muni.cz/www/moravec/c5966/}}

\author % (optional, for multiple authors)
{Zdeněk Moravec, hugo@chemi.muni.cz}

\date{}

\begin{document}
\maketitle

\pagebreak

\section{Průběh cvičení}

	\textbf{Návod není nutné tisknout!}

	Cvičení probíhá v laboratoři C12/112. Doba cvičení je 2--3 hodiny.

	\begin{enumerate}
	\item Krátký úvod k IR spektroskopii \textit{(A12/112)}
	\item Spuštění spektrometrů
	\item Změření IR spektra atmosféry, stanovení vlhkosti uvnitř přístroje
	\item Měření IR spekter vzorků v KBr tabletách a metodou ATR
	\item Interpretace IR spekter
	\end{enumerate}

\subsection{Protokol}

Protokol zašlete na adresu hugo@chemi.muni.cz \textit{do dvou týdnů} ode dne konání cvičení. Optimálním formátem je PDF.

\subsubsection{Doporučená struktura protokolu}

	\begin{enumerate}
	\item Hlavička (Jméno, datum konání cvičení)
	\item Princip
	\item Postup
	\item Spektra (naměřená spektra studenti dostanou v textovém formátu)
	\item Interpretace spekter
	\item Závěr
	\end{enumerate}

\newpage
\section{Infračervená spektroskopie}

\subsection{Stanovení vlhkosti uvnitř IR spektrometru}


Hodnota vlhkosti uvnitř spektrometru je důležitá, protože optika je citlivá na stopy vlhkosti. Pro stanovení vlhkosti nastavíme spektrometr následujícím způsobem:

\vspace{5mm}
\begin{tabular}{|c|l|}
	\hline
	Počet skenů (background) & 16 \\\hline
	Počet skenů (vzorek) & 1 \\\hline
	Rozlišení & 2~cm$^{-1}$ \\\hline
\end{tabular}
\vspace{5mm}

Po změření uložíme pozadí a odečteme hodnotu maximální intenzity (I$_{\textrm{MAX}}$) a hodnotu intenzity pásu 1559~cm$^{-1}$ (I$_{1559}$). Vlhkost pak vypočítáme:
\\
\\
$\textrm{M}_{\textrm{REL}} = (1 - \frac{\textrm{I}_{1559}}{\textrm{I}_{\textrm{MAX}}}) \cdot 100 \%$

\begin{figure}[h]
	\includegraphics[keepaspectratio,height=12cm]{img/moisture.png}
\end{figure}

\newpage

\subsection{Měření IR spekter vzorků v suspenzi v KBr tabletách}
	1--3~mg vzorku smícháme s cca 300~mg KBr a směs rozetřeme v~achátové třecí misce. Získaný prášek nasypeme do lisovací matrice a lisujeme pod tlakem 8--9 tun po dobu cca 1~minuty.

\begin{figure}[h]
	\includegraphics[keepaspectratio,height=15cm]{img/KBr.png}
\end{figure}
\newpage

\subsection{Měření IR spekter vzorků metodou ATR}
Vzorek nasypeme na krystal diamantu, přitlačíme hrotem a změříme spektrum. Vzorky není potřeba žádným způsobem upravovat.

\begin{figure}[h]
	\includegraphics[keepaspectratio,width=10cm]{img/atr.png}
\end{figure}

\subsection{Vyhodnocení}

Studenti dostanou naměřená IR spektra v textovém formátu, úkolem bude vytvořit grafický záznam spektra (doporučuji využít Gnuplot) a přiřadit nejintenzivnější pásy vibracím vazeb v molekule vzorku.

\end{document}
